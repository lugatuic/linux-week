\documentclass{beamer}

\usepackage{svg}

\usetheme{Boadilla}

% Title
\title{Package Management in Linux}
\subtitle{Linux Week \the\year{}}
\author{Soham S Gumaste}
\date{\today}
\institute{Linux Users Group @ UIC}

\begin{document}
\begin{frame}
	\titlepage
\end{frame}

\begin{frame}{Table of Contents}
	\tableofcontents[pausesections]
\end{frame}

\section{Definitions}
\begin{frame}{What is a Package?}
	Most \textbf{software applications} designed for \underline{Linux}
	systems are distributed as \texttt{packages}, which are archives that
	contain pre-compiled \textbf{binary software files}, installation
	scripts, \textbf{configuration files}, \textbf{dependency
	requirements}, and other details about the software.
\end{frame}

\begin{frame}{What is a Dependency?}
	Simply put, if a package needs a second package to work, the second
	package is a dependency of the first.
	
	\pause

	Think of them like \textbf{course prerequisites}!
\end{frame}

\begin{frame}{Repository}
	A collection of \textbf{packages} provided to you to install software.
\end{frame}

\begin{frame}{Why should I care?}
	How many of you have installed something from a random website and have
	it break your computer?! (Show of hands)

	\pause

	Packages installed from package managers are \textbf{vetted} by your
	distribution (Ubuntu, Fedora etc) and will almost always work as
	expected and NOT break your system!

	\pause

	Packages are also \textbf{cryptographically signed} to ensure you never
	get non-genuine packages.
\end{frame}

\section{Package Management in Ubuntu}
\begin{frame}{Table of Contents}
	\tableofcontents[currentsection]
\end{frame}

\begin{frame}{Package Management in Ubuntu}
	Ubuntu uses a tool called \texttt{apt} to manage packages.

	Apt will reach out to Ubuntu servers and download, verify, and install
	packages you ask for.

	\begin{exampleblock}{Example}
		\texttt{sudo apt install cmatrix}

		\texttt{sudo} is needed as installing packages needs root privileges!
	\end{exampleblock}
\end{frame}

\begin{frame}{Important}
	Always read all apt prompts and answer them accordingly!
	(Google "LinusTechTips yes do as I say")
\end{frame}

\begin{frame}{Common Apt Commands}
	\begin{enumerate}
		\item Install a new package \texttt{sudo apt install
			<package\_name>}
		\item Search for a package in the repositories \texttt{apt
			search <search\_term>} Note that \texttt{search\_term}
			can be a \underline{RegEx}!
		\item Remove a package \texttt{sudo apt remove <package\_name>}
		\item Update all packages \texttt{sudo apt update} then
			\texttt{sudo apt upgrade}
	\end{enumerate}
\end{frame}

\begin{frame}{Cool Packages}
	\begin{itemize}
		\item CMatrix
		\item cowsay
		\item sl
		\item Figlet
		\item fortune
		\item xeyes
		\item aafire
		\item asciiquarium
		\item rig
	\end{itemize}
\end{frame}

\begin{frame}{Closing Remarks}
	\begin{center}
		\Huge Thank you!
	\end{center}
\end{frame}

\begin{frame}{Closing Remarks}
	\begin{columns}
		\begin{column}{0.5\textwidth}
			\textbf{Officers}
			\begin{figure}
				\centering
				\includegraphics[width=0.60\textwidth]{officers.png}
			\end{figure}
		\end{column}
		\begin{column}{0.5\textwidth}
			The information in this presentation will be made
			available\footnotemark on our website!\\
			\url{https://lug.cs.uic.edu}
			
			\bigskip
			Join our Discord!

			\begin{figure}
				\centering
				\includesvg[width=0.5\textwidth]{lug-discord.svg}
				\caption{\url{https://discord.gg/NgxTR7PX5e}}
			\end{figure}
		\end{column}
	\end{columns}

	\footnotetext{sooner or later}
\end{frame}

\end{document}

% vim: set tw=80 ts=4 sw
